\documentclass[a4paper,10pt]{article}

\usepackage[dvipsnames]{xcolor}
\usepackage[margin=1in]{geometry}

\title{Quiz 5}
\author {Victoria Deng} 
\date{\small October 2020} 

\begin{document}
\maketitle 
\section*{Virtual Reality}
\textcolor{Orchid}{\textbf{Virtual Reality:}} A series of inputs and outputs that embodies a sensory experience of a virtual environment\\\\
\textcolor{Orchid}{\textbf{Applications:}} Used for education, entertainment, industrial design etc. \\\\
\textcolor{Orchid}{\textbf{Gorilla Arm:}} Holding up arms for extended periods of time (over 90 seconds) can cause fatigue. (Known as gorilla arm effect)\\\\
\textcolor{Orchid}{\textbf{Gun slinging:}} An alternative that keeps hands at sides, and you use your fingers as inputs. We lose hand-to-display matching, but it does reduce fatigue.\\\\
\textcolor{Orchid}{\textbf{Object Manipulation:}} Hands can be used to interact with a virtual environment, however it is difficult to track a hand and object together. Some solutions are gloves (unhygenic), applications that don't require contact, or using instrumented environments.\\\\
\textcolor{Orchid}{\textbf{Gesturing:}} Not natural, and can be awkward. Should either assume that users want to learn new gestures (questionable), or restrict it to universal gestures (pinching etc)\\\\
\textcolor{Orchid}{\textbf{Body Misalignment:}} A virtual and physical body will never be perfectly aligned in time and space. Users cannot rely on their senses as they normally would, as it is not synchronised.\\\\
\textcolor{Orchid}{\textbf{Text Entry:}} VR doesn't perform well in tasks that require fast and precise motor control, could use gestural entry but it is still much slower than typing. Sign language can improve wpm, but requires learning. \\\\
\textcolor{Orchid}{\textbf{Motion Sickness:}} A recurring issue since VR prototype. \\\\
\textcolor{Orchid}{\textbf{Field of View:}} The field of view of a head mounted display is much narrower than your real peripheral vision. Users often lose visual context and need to glance around to orient themselves.\\\\

\textcolor{Orchid}{\textbf{}}\\\\
\textcolor{Orchid}{\textbf{}}\\\\
\textcolor{Orchid}{\textbf{}}\\\\

\end{document} 
